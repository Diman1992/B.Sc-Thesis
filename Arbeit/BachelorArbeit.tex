\documentclass[11pt,a4paper,twoside]{report}
\input{Tex/longheader.tex}
\input{Tex/global}
 \setlength{\parskip}{1.5ex}
\begin{document}
\begin{spacing}{1,5}
\pagenumbering{Roman}

% Anmerkung: Die Seitenraender wurden asymmetrisch gewaehlt,
%            damit genug Platz fuer eine Klemmbindung da ist.
%            Da neue Kapitel auf der rechten Seite (ungerade
%            Seitennummer) beginnen sollten, muss ggf. am Ende
%            des vorhergehenden Kapitels eine Leerseite
%            eingefuegt werden:
%
%            \newpage
%            \thispagestyle{empty}
%            \ \\
%            \newpage
%
%            Die Seitenraender koennen aber auch in der Datei Tex/global.tex
%            veraendert werden.

% >>> Titelseite <<<

\newcommand{\thetitle}{Formfaktoren des semileptonischen $D^+ \rightarrow \bar K^0 l^+ \nu$ Zerfalls}

\thispagestyle{empty}
\begin{center}
\Huge\textbf{\thetitle}
\vfill
% Note that the size is given in normal parentheses
% instead of curly brackets.
% Define external vertices from bottom to top
\vfill
\Large
Bachelorarbeit \\ zur Erlangung des akademischen Grades \\ Bachelor of Science \\
\vspace{20pt}
\normalsize
vorgelegt von \\[5pt]
{\Large Dimitrios Skodras} \\[5pt]
geboren in Aschaffenburg \\
\vspace{20pt}
Lehrstuhl für Theoretische Physik IV \\ Fakultät Physik \\
Technische Universität Dortmund \\ 2014
\end{center}
\newpage

% >>> Gutachterseite <<<

\thispagestyle{empty}
\vspace*{\fill}
\begin{tabbing}
1. Gutachter : \=\kill
1. Gutachter : \>Prof. Dr. Musterfrau \\[11pt]
2. Gutachter : \>Prof. Dr. Mustermann \\[11pt]
\end{tabbing}
\vspace{11pt}
Datum des Einreichens der Arbeit: TT.\,Monat\,JJJJ
\newpage

% >>> Kurzfassung/Abstract <<<

\thispagestyle{empty}
%Kuzfassung
\section*{Kurzfassung}
Hier folgt eine kurze Zusammenfassung des Inhalts und der Ergebnisse der Arbeit in deutscher Sprache.\\
\ \\

\newpage

% >>> Hauptteil <<<

\addcontentsline{toc}{chapter}{Inhaltsverzeichnis}
\tableofcontents\newpage
\addcontentsline{toc}{chapter}{Abbildungsverzeichnis}
\listoffigures\newpage
\addcontentsline{toc}{chapter}{Tabellenverzeichnis}
\listoftables\newpage

\setcounter{page}{0}
\pagenumbering{arabic}

\chapter{Einleitung}


\chapter{Theorie}
Ehe die Matrixelemente errechnet werden, ist es erforderlich, Kenntnis von beteiligten Zusammenhängen zu haben. Zum einen werden die beim Zerfall beteiligten
Teilchen und ihre fundamentalen Wechselwirkungen beleuchtet. Dazu gehören die D- und K-Mesonen mit Quarks als Konstituenten, sowie den Leptonen als Teilchen
und die schwache Wechselwirkung 
Da es sich um sehr kleine Teilchen handelt ist eine Betrachtung der Quantenmechanik erforderlich. 
Da sie zusätzlich leicht und daher hinsichtlich der Lichtgeschwindigkeit schnell sind, bleiben Gesetzmäßigkeiten der relativistische Kinematik nicht aus.
Zum anderen werden allgemein Formfaktorparametrisierungen und im speziellen die Reihenentwicklung thematisiert.
\section{Voraussetzungen moderner Physik}
Die physikalischen Errungenschaften aus der ersten Hälfte des 20. Jahrhunderts stellen aufgrund ihrer Richtigkeit und Exaktheit die Voraussetzungen an moderne 
Theorien, dass diese immer gelten müssen. Hiermit sind die spezielle Relativitätstheorie und die Quantenmechanik gemeint, sowie ihre Vereinigung, die relativistische
Quantenmechanik gemeint. Davon sind im Rahmen dieser Arbeit die relativistische Kinematik, die Dirac-Gleichung sowie die störungstheoretische ``Goldene Regel'' von Fermi
von Bedeutung.

\subsection{Relativistische Kinematik}
Die SRT stellt die fundamentale Forderung, dass die Form der Naturgesetze unabhängig vom Inertialsystem gleich ist. Mit der Lichtgeschwindigkeit $c$ als
größte vorkommende Geschwindigkeit ist sie ebenfalls in allen Inertialsystemen gleich groß. Die relativistische Energie-Impuls Beziehung 
$E^2 = \left(mc^2\right)^2 + \left(\vec pc\right)^2$ beschreibt einen allgemeinen Zusammenhang zwischen der Energie $E$, der Masse $m$ und dem Impuls $\vec p$.
In der Hochenergiephysik ($E_{\text{CMS}} \approx 10^4$ GeV) gilt der hochrelativistische Grenzfall, bei dem Energie hauptsächlich durch den Impuls bestimmt wird.
In natürlichen Einheiten wird $c = 1$ gesetzt, was zu $E^2 = \left|p\right|^2$ führt.

Zur Beschreibung der Bewegung von relativistischen Teilchen wird wegen der Energie-Impuls-Beziehung und der Verknüpfung von Raum und Zeit ($x = t$) das Konzept
der Vierer-Vektoren eingeführt. Es gestaltet sich so, dass die Zeit als 0. Komponente des Raums und die Energie als 0. Komponente des Impulses angesehen werden
kann, was die 4-Dimensionalität zeigt \cite{RelKin}.
\begin{align}
 x^\mu &= (t,\,x,\,y,\,z)^\mu \hspace{2cm} \text{Vierer-Ort}\\
 p^\mu &= (E,\,p_x,\,p_y,\,p_z)^\mu \hspace{1.35cm} \text{Vierer-Impuls}
\end{align}
Der Index $\mu$ kann ganzzahlige Werte zwischen 0 und 4 annehmen und steht für die jeweilige Komponente des Vektors. Im Gegensatz zu euklidischen Räumen 
kann ein Skalarprodukt zweier Vierer-Vektoren nur dann beschrieben werden, wenn einer kovariant (Index unten) und der andere kontravariant (Index oben) ist.
Diese Überführung geschieht durch die Minkowskimetrik, die die Norm unter Lorentz-Transformationen konstant lässt
\begin{equation}
  p^2 = p^\mu \eta_{\mu \nu} p^\nu = p^\mu p_\mu = E^2 - \vec p^2 = m^2,
\end{equation}
was wieder die relativistische Energie-Impuls-Beziehung ist. Die Einsteinsche Summenkonvention wird hierbei angewandt.
\subsection{Dirac-Gleichung}
\vspace{-0.5cm}
Aus der nicht-relativistischen Schrödinger-Gleichung als quantenmechanische Wellengleichung ergibt sich durch erste Quantisierung \cite{TeilFortgeschr} eine Ersetzung von Energie 
und Impuls durch partielle Differentialoperatoren
\begin{align*}
 E \rightarrow \text{i}\hbar\partial_t, \quad p \rightarrow -\text{i}\hbar\nabla.
\end{align*}
Da die Schrödinger-Gleichung nicht lorentzinvariant ist, sind andere Ansätze durchgeführt worden, die der zuvor genannten Energie-Impuls-Beziehung $p^\mu p_\mu = m^2$
genügen. Die Klein-Gordon-Gleichung für spinlose Teilchen, die aus ihr direkt folgt, ist zwar relativistisch korrekt, weist jedoch keine positiv definite 
Wahrscheinlichkeitsdichte auf. Die Dirac-Gleichung für Spin$\frac12$-Teilchen ist ebenfalls unter Lorentztransformationen invariant und besitzt nun
zusätzlich eine positiv definite Wahrscheinlichkeitsdichte, was eine Interpretation ihrer Lösungen als Wahrscheinlichkeitsamplitude zulässt. Daher muss diese
Gleichung linear in der ersten Zeit- und Ortsableitung sein und der folgenden Schrödinger-Form genügen \cite{RelQuantMech}
\begin{equation}
 \text{i} \partial_t \psi = \left(-\text{i}\alpha^k\partial_k + \beta m\right)\psi = \mathcal{H} \psi,
 \label{eq_diracSchroedinger}
\end{equation}
mit $\hbar = 1$ und $H$ dem diracschen Hamiltonian. $\alpha^k = \gamma^0\gamma^k$ und $\beta = \gamma^0$ sind die historischen Dirac-Matrizen. Gelöst wird
diese Schrödinger-Form von der folgenden Dirac-Gleichung
\begin{equation}
 (\slashed{p} - m )\psi = 0,
 \label{eq_dirac}
\end{equation}
mit $\slashed{p}$ als Impulsoperator in der Feynman-Slash-Notation
\begin{equation}
 \slashed{A} := \gamma^\mu A_\mu = \gamma^0 A_0 - \gamma^i\cdot A_i
\end{equation}
und $\psi$ als Wellenfunktion, da $\gamma^\mu \in \text{M}^{4\text{x}4}$, mit den vier Freiheitsgraden: Teilchen, Antiteilchen, Spin-Up, Spin-Down.
Die $\gamma$-Matrizen in der Dirac-Pauli-Notation lauten
\begin{equation*}
  \gamma^0 = \begin{pmatrix} 
              \mathbbm{1} & 0 \\
              0 & -\mathbbm{1}
             \end{pmatrix},\quad \gamma^i = \begin{pmatrix}
					0 & \sigma^i\\
					-\sigma^i & 0
				      \end{pmatrix} \quad \text{und}\quad \gamma^5 = \begin{pmatrix}
									  0 & \mathbbm{1}\\
									  \mathbbm{1} & 0
									   \end{pmatrix}
\end{equation*}
mit $\mathbbm{1}$ als 2x2-Einheitsmatrix und den $\sigma^i$ als Paulimatrizen
\begin{equation*}
 \sigma^1 = \begin{pmatrix}
             0 & 1\\
             1 & 0
            \end{pmatrix},\quad \sigma^2 = \begin{pmatrix}
					    0 & -\text{i}\\
					    \text{i} & 0
					    \end{pmatrix}\quad \text{und} \quad\sigma^3 = \begin{pmatrix}
									    1 & 0\\
									    0 & -1
									    \end{pmatrix}.									    
\end{equation*}
Die Lösungen der freien Dirac-Gleichung sind folgende Wellenfunktionen
\begin{align}
 \psi_+(x) = u(p)^{(1,2)} \e^{-ip^\mu x_mu}\qquad\text{und}\qquad \psi_-(x) = v(p)^{(1,2)} \e^{+ip^\mu x_mu},
 \label{eq_diracLsg}
\end{align}
die eingesetzt in \eqref{eq_dirac} die Gleichungen für die Spinoren
\begin{align}
 (\slashed{p} - m )u^{(1,2)} &= \bar u^{(1,2)}(\slashed{p}-m) = 0 \\
 (\slashed{p} + m )v^{(1,2)} &= \bar v^{(1,2)}(\slashed{p}+m) = 0,
\end{align}
ergeben, wobei $\slashed{p}$ hier nun kein Operator, sondern der Eigenwert des Viererimpulses der ebenen Welle \eqref{eq_diracLsg} ist und $\bar a = a^\dagger \gamma^0$,
mit $\dagger$ als komplexe Konjugation und Transposition. Lösungen dieser Bispinoren der Teilchen ($u$) und Antiteilchen ($v$), die nur von Impuls und Spin abhängen,
lauten nun
\begin{equation}
 u(p,s)^{(1,2)} = \mathcal{N} \begin{pmatrix}
                          \chi^{(1,2)}\\
                          \frac{\vec \sigma \vec p}{E+m} \chi_s^{(1,2)}
                         \end{pmatrix} \qquad \text{und}\qquad v(p,s)^{(1,2)} = \mathcal{N} \begin{pmatrix}
                          \frac{\vec \sigma \vec p}{E+m} \chi_s^{(2,1)}\\
                          \chi^{(2,1)}                          
                         \end{pmatrix},
\end{equation}
mit einer Normierung $\mathcal{N}$, die für gewöhnlich $\sqrt{E + m}$ gewählt wird und dem nicht-relativistischen Spinor für Spin $\frac12$-Teilchen $\chi_s$.
Die Energieeigenwerte der freien Lösungen für Teilchen $\psi_+$ sind positiv und die der Antiteilchen $\psi_-$ negativ, was mit der Löchertheorie interpretiert
wird. Wie bereits erwähnt, existiert bei der Dirac-Gleichung ein Wahrscheinlichkeitsstrom $j^\mu$, der der lorentzinvarianten Kontinuitätsgleichung
\begin{equation}
 \partial_\mu j^\mu = 0
\end{equation}
genügt. Zur Ermittlung wird \eqref{eq_diracSchroedinger} von links mit der komplex konjugierten Wellenfunktion und die komplex konjugierte
Form von \eqref{eq_diracSchroedinger} von rechts mit der normalen Wellenfunktion multipliziert. Hierzu dient die Gleichheit 
$(\gamma^\mu)^\dagger = \gamma^0\gamma^\mu\gamma^0$. Die daraus resultierenden Gleichungen voneinander abgezogen ergeben die eben genannte Kontinuitätsgleichung.
Der Wahrscheinlichkeitsstrom lautet somit
\begin{equation}
 j^\mu = \bar \psi \gamma^\mu \psi.
 \label{eq_diracstrom}
\end{equation}


\subsection{Fermis Goldene Regel}
\label{sec_fermi}
In der Elementarteilchenphysik werden unter anderem Zerfälle untersucht, dessen Edukte sehr kurze Lebenszeiten $\tau$ haben und diese sich daher nicht sehr präzise
bestimmen lassen. Mit der Heisenbergschen Unschärferelation gelingt es, eine sogenannte Zerfallsbreite $\Gamma$ als bestimmbare Größe zu erheben
\begin{equation}
 \Gamma \tau = 1 \quad \leftrightarrow\quad  \Gamma = \frac{1}{\tau},
\end{equation} 
die ein Maß für die Breite eines gemessenen Peaks darstellt. Für einen beliebigen Zerfall ist es von Interesse, eben diese Zerfallsbreite auszurechnen und dazu
benötigt man im Allgemeinen etwas wie eine Zerfallsamplitude $M$ (auch Matrixelement genannt) und einen verfügbaren lorentzinvarianten Phasenraum $\Phi$ 
\cite{Griffiths}\cite{TeilFortgeschr}. Die Amplitude,
die mithilfe der Feynman Regeln berechenbar ist, enthält hierbei die dynamischen Informationen, der Phasenraum die kinematischen, also die Massen, Energien
und Impulse der beteiligten Teilchen. Nach Fermis goldener Regel lässt sich die differentielle Zerfallsbreite ausdrücken durch
\begin{equation}
 \dx \Gamma(D \rightarrow K l \nu) = \frac{\left|M\right|^2}{2m_D}\dx \Phi(K, \,l,\, \nu),
 \label{eq_fermirule}
\end{equation}
mit $m_D$ der Masse des D-Mesons. Der Phasenraum lässt sich schreiben als
\begin{equation}
 \dx \Phi = (2\pi)^4 \frac{\dx^3p_K}{2(2\pi)^3E_k}\frac{\dx^3k_l}{2(2\pi)^3E_l}\frac{\dx^3k_\nu}{2(2\pi)^3E_\nu}\delta(p_D-p_K-k_l-k_\nu),
\end{equation}
wobei $k_i$ fortan Leptonimpulse, $p_i$ Hadronimpulse, die Summe $p_D+p_K$ knapp als $P$ und die Differenz $p_D - p_K$ als $q$ bezeichnet werden. 
Dabei ist $\delta$ die diracsche Deltafunktion, deren Aufgabe die Erhaltung von Energie und Impuls ist. Ziel ist es nun grob einen Ausdruck zu finden,
theoretisch die differentielle Zerfallsbreite zu berechnen. Das Matrixelement berechnet sich über den Erwartungswert eines Hamiltonians,
der sich in einen Quark- und einen Leptonenstrom aufteilen lässt, die wiederum mit \eqref{eq_diracstrom} umgeformt werden können. 
Ihr Aussehen wird später in Abschnitt \ref{sec_schwacheWW} näher behandelt. 
\begin{align}
 M &= \big\langle Kl\nu|\mathcal{H}|D\big\rangle\\
 &= \big\langle K\, \big|j_\text{quark}^\mu\big|\,D \big\rangle \,\cdot\,\big\langle l\nu\,\big|j_\mu^\text{lepton}\big|\,0\big\rangle\nonumber\\
 &= \big\langle K(p_K)\, \big|\bar s \gamma^\mu(1-\gamma_5) c \big|\,D(p_D) \big\rangle \, \cdot \,\bar u(k_l) \gamma_\mu(1-\gamma_5)v(k_\nu)\nonumber\\
 &=\frac{G_F V}{\sqrt{2}} \big[f_+(q^2) P^\mu  + f_-(q^2) q^\mu\big] \bar u \gamma_\mu(1-\gamma_5)v\nonumber
 \end{align}
Im Vorfaktor gehen die Fermikonstante $G_F$ und ein ebenfalls in \ref{sec_schwacheWW} wiederkommendes Matrixelement $V$ ein. Weiterhin sind $f$ die noch zu
diskutierenden Formfaktoren, von denen an dieser Stelle nur $f_+$ weiter betrachtet wird. Nach der Quadratur von $M$ wird der leptonische Anteil über 
Casimirs Trick \cite{Griffiths} umgeformt
\begin{align}
 \big|M\big|^2 &= \frac{G_F^2|V|^2}{2}|f_+(q^2)|^2 P^\mu P^\nu \cdot 8\big(k_{l,\mu} k_{\nu,\nu} - g_{\mu\nu}k_lk_\nu + k_{l,\nu}k_{\nu,\mu}\big)\nonumber\\
 &=4G_F^2|V|^2 |f_+(q^2)|^2 \big(2P^\mu P^\nu - P^2 g^{\mu\nu}\big) k_{l,\nu}k_{\nu,\mu},
 \label{eq_fermiMelement}
\end{align}
wobei die Indizes $k_{\text{lepton},\text{lorentzindex}}$ bedeuten. Für die Phasenraumbetrachtung sind im Ruhesystem des D-Mesons die Gleichheiten
\begin{align}
 \frac{\dx^3 p_K}{2E_K} = 2\pi |p_K| \dx E_K \qquad \text{und} \qquad |p_K| = \frac{\sqrt{\lambda(m_D^2,m_K^2,q^2)}}{2m_D},
\end{align}
sowie die Integration
\begin{align}
 \int \frac{\dx^3k_l}{2E_l}\frac{\dx^3k_\nu}{2E_\nu}\delta^4(q-k_l-k_\nu)k_\mu k_\nu = \frac{\pi}{24}(q^2 g_{\mu\nu} + 2 q_\mu q_\nu)
\end{align}
gegeben. Das ergibt sich für das Phasenraumvolumen mit Übernahme von $k_{l,\nu}$ und $k_{\nu,\mu}$ aus \eqref{eq_fermiMelement} somit
\begin{align}
 \dx \Phi &= (2\pi)^4 \frac{\dx p_k}{(2\pi)^9 2E} \int \frac{\dx^3k_l}{2E_l}\frac{\dx^3k_\nu}{2E_\nu}\delta^4(q-k_l-k_\nu)k_\mu k_\nu \nonumber\\
 &= \frac{|p_K| \dx E}{(2\pi)^4}\frac{\pi}{24}(q^2g_{\mu\nu} + 2 q_\mu q_\nu)
 \label{eq_fermiPhase}
\end{align}
Um nun schließlich die differentielle Zerfallsbreite zusammenzufassen, werden \eqref{eq_fermiMelement} und \eqref{eq_fermiPhase} angepasst verwendet. Im
folgenden Ausdruck wird eine Gleichheit
\begin{align}
 \big(2P^\mu P^\nu - P^2 g^{\mu\nu}\big)\big(2q_\mu q_\nu + q^2g_{\mu\nu}) = 4 \lambda(m_D^2,m_K^2,q^2) = 16 m_D^2 |p_K|^2
\end{align}
und ein aus der Kinematik ableitbarer Zusammenhang $\dx E = \dx q^2 / 2m_D$ benutzt. Die zu Beginn genannte Gleichung \eqref{eq_fermirule} wird 
nun abschließend ausgedrückt als
\begin{align}
 \dx \Gamma = \frac{G_F^2|V|^2}{24\pi^3} |f_+(q^2)|^2 |p_K|^3 \dx q^2.
\end{align}
Da im weiteren Verlauf $|f_+(q^2)|V$ berechnet werden soll, jedoch die komplette von $q^2$ abhängige Zerfallsbreite gemessen wird, ist dieser eben hergeleitete
Zusammenhang dieser Größen von Bedeutung.



\section{Stadardmodell der Elementarteilchenphysik}
Das SM setzt sich aus zwei definierenden Eigenschaften zusammen, den Teilchen und den Eichtheorien, die diese beschreiben. Die sichtbare Materie wird 
aus Fermionen zusammengesetzt. Zu den Quantenfeldtheorien des SMs gehören die Quantenelektrodynamik (QED), die Quantenchromodynamik (QCD) und die schwache
Wechselwirkung, die hier näher beleuchtet wird. 

\subsection{Teilcheninhalt}
Seit langem ist bekannt, dass die als unteilbar angenommenen Atome aus Konstituenten bestehen. Die Elektronenhülle und den Atomkern, der seinerseits aus
Protonen und Neutronen zusammengesetzt ist, die ihrerseits wiederum aus Quarks bestehen. Nach derzeitigem Stand gelten Elektronen und die anderen geladenen 
Leptonen $l$ und Quarks als punktförmige
Teilchen, die keine Substruktur aufweisen. Antiteilchen gibt es zu jedem Teilchen. Sie gleichen sich zwar in ihrer Masse, tragen jedoch in allen ladungsartigen Quantenzahlen, wie der Leptonenzahl oder
elektrische Ladung und Parität ein entgegengesetztes Vorzeichen. Die stabilen, sehr leichten Neutrinos $\nu$ existieren zwar nicht in gebundenen Zuständen, gelten jedoch als elementar und sind daher 
ebenfalls im elementaren Teilchenzoo \cite{PDG} in Tabelle \ref{tab_particlezoo} aufgeführt.
\begin{table}[H]
\begin{tabular}{c|cccc|ccc} \toprule 
 Generation & & $m$ in MeV & $\tau$ in s & $q$ in e & & $m$ in MeV & $q$ in e\\
 \midrule
  1 & $e$ & 0,511 & stabil & -1 & $u$ & 1,5-4 & +$\frac23$\\
  &$\nu_\text{e}$& <10$^{-6}$ &  & 0 & $d$ & 4-8 & -$\frac13$\\
  2 & $\mu$ & 105,7 & 2,2$\cdot 10^{-6}$ & -1 & $c$ &1150-1350& +$\frac23$\\
  &$\nu_\mu$ & & & 0 & $s$ &80-135& -$\frac13$\\
  3& $\tau$ &1777& 2,9$\cdot 10^{-13}$ & -1 & $t$ & 169000-174000 & +$\frac23$\\
  &$\nu_\tau$& & & 0 & $b$ &4100-4400 & -$\frac13$
\\\bottomrule \bottomrule
 \end{tabular}
\caption{Kenndaten elementarer Fermionen (ohne Antiteilchen)}
\label{tab_particlezoo}
\end{table}
\noindent
Aus Quarks $q$ und ihren Partnern, den Antiquarks $\bar q$ ist es nun möglich, verschiedenste Kombinationen zu bilden, die jedoch nach außen hin immer eine
ganzzahlige Ladung tragen müssen. Diese Quarkkombinationen werden Hadronen genannt und werden durch die Anzahl an Valenzquarks in (Anti-)Baryonen ($\bar q\bar q\bar q$/$qqq$) und in 
Mesonen ($q\bar q$) unterklassifiziert. Hier interessant sind die Mesonen, die je aus mindestens einem Quark der 2. Generation bestehen. Für diese Arbeit
relevant sind das $D^+$-Meson, welches unter schwacher Wechselwirkung in ein $\bar K^0$-Meson übergeht. In Tabelle \ref{tab_DKMeson} sind ihre Attribute
aufgeführt.
\begin{table}[H]
\begin{tabular}{c|ccc|cccc} \toprule 
  & $q\bar q$ &  $m$ in MeV & $q$ in e & $I(J^P)$ & $I_z$ & $S$ & $C$\\
 \midrule
  $D^+$ & $c\bar d$ & 1869  & +1 & $\frac12(0^-)$ & +$\frac12$ & 0 & 1\\
 $\bar K^0$ & $s\bar d$ & 497  & +0 & $\frac12(0^-)$ & +$\frac12$ & -1& 0
\\\bottomrule \bottomrule
 \end{tabular}
\caption{Kenndaten der im Zerfall beteiligten Mesonen}
\label{tab_DKMeson}
\end{table}
\noindent
Die hier neu auftauchenden Größen sind im einzelnen der starke Isospin $I$, seine dritte Komponente $I_z$, der Gesamtdrehimpuls $J$, die Parität $P$, die
Strangeness $S$, sowie die Charmeness $C$. Da je ein Quark der ersten Generation vertreten ist, nimmt $I$ den Wert $\frac12$ an. Da es sich hierbei um ein 
$\bar d$ handelt, ist $I_z$ $+\frac12$. Mesonen tragen ganzzahligen Gesamtdrehimpuls und da hier die Spinrichtungen ihrer Valenzquarks entgegengesetzt 
ausgerichtet sind, verschwindet dieser. Weil $D^+$ und $\bar K^0$ pseudoskalar sind, also unter Raumspiegelung ihr Vorzeichen ändern, ergibt sich $P=-1$.

\subsection{Schwache Wechselwirkung}
\label{sec_schwacheWW}
parität
V-A-Theorie
vierstromwechselwirkung
ckm-matrix


\section{Parametrisieren von Formfaktoren}
motivation von formfaktoren
f+ P1 + f- P2 (und f0 P3)
z-expansion

\chapter{Ergebnisse}
\section{Kinematische Größen}
Die Formfaktoren werden allgemein in Abhängigkeit des Impulsübertrags $q^2$ angegeben. 
\begin{figure}[H]
\includegraphics[width=1\textwidth]{Abbildungen/DZerfall.png}
\caption{Zwei Zerfallsmöglichkeiten des $D^+$-Mesons  mit extremen $q^2$-Werten}
\label{pic_DZerfall}
\end{figure}


\section{Ermittlung der Formfaktoren}
\subsection{Die Formfaktoren $f_0$ und $f_-$}
\subsection{Fit des Formfaktors $f_+$}


\chapter{Zusammenfassung und Ausblick}

Hier sollen die Ergebnisse zusammengefasst und weiterf\"uhrende Untersuchungen diskutiert werden. 


% >>> Anhang <<<

\begin{appendix}
%\input{Kapitel/Anhang}
\end{appendix}

% >>> Literaturverzeichnis <<<

\renewcommand{\bibname}{Quellenverzeichnis}
\addcontentsline{toc}{chapter}{\bibname}
%\bibliographystyle{unsrt} 
%\bibliography{BachelorArbeit}
\begin{thebibliography}{xxx}
 \bibitem[1]{RelKin}Nedden zur, M.: \textit{Detektoren der Elementarteilchenphysik}[pdf]\\ \href{http://www-hera-b.desy.de/people/nedden/lectures/05_06/dettph/dettph_cont.pdf}{http://www-hera-b.desy.de/people/nedden/lectures/05\_06/dettph/dettph\_cont.pdf}, 2006
 \bibitem[2]{TeilFortgeschr}Schleper, P.: \textit{Teilchenphysik für Fortgeschrittene}[pdf]\\ \href{http://www.desy.de/~schleper/lehre/}{http://www.desy.de/\midtilde schleper/lehre/}, 2011
 \bibitem[3]{RelQuantMech}Bjorken, J.D., Drell, S.D.: \textit{Relativstic Quantum Mechanics}, 1964, \\ISBN-13 978-0072320022 
 \bibitem[4]{Griffiths}Griffiths, D.: \textit{Introduction to Elementary Particles}, 2008, \\ISBN-13 978-3527406012
 \bibitem[5]{PDG} Beringer, J. et al.: \textit{Particle Data Group}, Phys. Rev. D86, 010001, 2012
 \bibitem[6]{DissForm}Offen, N.: \textit{B-Zerfallsformfaktoren aus QCD-Summenregeln}\\ \href{http://d-nb.info/987811061}{http://d-nb.info/987811061}, 2008
 \bibitem[30]{bla}{Versuch V28 Elektronen-Spin-Resonanz }
\end{thebibliography}


\newpage
\thispagestyle{empty}
\ \\

% >>> Erklaerung <<<

\input{Kapitel/Erklaerung}
\end{spacing}

\end{document}