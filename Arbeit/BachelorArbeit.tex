\documentclass[11pt,a4paper,twoside]{report}
\input{Tex/longheader.tex}
\input{Tex/global}
\begin{document}
\pagenumbering{Roman}

% Anmerkung: Die Seitenraender wurden asymmetrisch gewaehlt,
%            damit genug Platz fuer eine Klemmbindung da ist.
%            Da neue Kapitel auf der rechten Seite (ungerade
%            Seitennummer) beginnen sollten, muss ggf. am Ende
%            des vorhergehenden Kapitels eine Leerseite
%            eingefuegt werden:
%
%            \newpage
%            \thispagestyle{empty}
%            \ \\
%            \newpage
%
%            Die Seitenraender koennen aber auch in der Datei Tex/global.tex
%            veraendert werden.

% >>> Titelseite <<<

\newcommand{\thetitle}{Formfaktoren des semileptonischen $D \rightarrow K l \bar \nu$ Zerfalls}

\thispagestyle{empty}
\begin{center}
\Huge\textbf{\thetitle}
\vfill
\begin{fmffile}{s2imple}
\begin{fmfgraph}(140,125)
% Note that the size is given in normal parentheses
% instead of curly brackets.
% Define external vertices from bottom to top
\fmfleft{i1,i2}
\fmfright{o1,o2}
\fmf{fermion,label = $u$}{i1,v1,o1}
\fmf{fermion}{i2,v2,o2}
\fmf{photon}{v1,o2}
\end{fmfgraph}
\end{fmffile}
\vfill
\Large
Bachelorarbeit \\ zur Erlangung des akademischen Grades \\ Bachelor of Science \\
\vspace{20pt}
\normalsize
vorgelegt von \\[5pt]
{\Large Dimitrios Skodras} \\[5pt]
geboren in Aschaffenburg \\
\vspace{20pt}
Lehrstuhl für Theoretische Physik IV \\ Fakultät Physik \\
Technische Universität Dortmund \\ 2014
\end{center}
\newpage

% >>> Gutachterseite <<<

\thispagestyle{empty}
\vspace*{\fill}
\begin{tabbing}
1. Gutachter : \=\kill
1. Gutachter : \>Prof. Dr. Musterfrau \\[11pt]
2. Gutachter : \>Prof. Dr. Mustermann \\[11pt]
\end{tabbing}
\vspace{11pt}
Datum des Einreichens der Arbeit: TT.\,Monat\,JJJJ
\newpage

% >>> Kurzfassung/Abstract <<<

\thispagestyle{empty}
%Kuzfassung
\section*{Kurzfassung}
Hier folgt eine kurze Zusammenfassung des Inhalts und der Ergebnisse der Arbeit in deutscher Sprache.\\
\ \\

\newpage

% >>> Hauptteil <<<

\addcontentsline{toc}{chapter}{Inhaltsverzeichnis}
\tableofcontents\newpage
\addcontentsline{toc}{chapter}{Abbildungsverzeichnis}
\listoffigures\newpage
\addcontentsline{toc}{chapter}{Tabellenverzeichnis}
\listoftables\newpage

\setcounter{page}{0}
\pagenumbering{arabic}

\chapter{Einleitung}

Hier folgt eine kurze Einleitung in die

\chapter{Theoretische Grundlagen}
\section{Stadardmodell der Elementarteilchenphysik}
ggf. über gruppen
\subsection{Elementarer Teilchenzoo}
elementare Teilchen
d-meson, k-meson
\subsection{Quantenchromodynamik}
parität

\subsection{Elektroschwache Wechselwirkung}
V-A-Theorie
vierstromwechselwirkung
ckm-matrix

\section{Relativistische Kinematik}
SRT
lorentzinvarianz
\section{Parametrisierung von Formfaktoren}
parametrisierungen nennen und auf den speziellen näher eingehen
AP1 + BP2
z-expansion

\cite{b:feynman}
\chapter{Messungen}
\section{Energiebereich von $q^2$}
\section{Ermittlung der Formfaktoren}


\chapter{Zusammenfassung und Ausblick}

Hier sollen die Ergebnisse zusammengefasst und weiterf\"uhrende Untersuchungen diskutiert werden. 


% >>> Anhang <<<

\begin{appendix}
%\input{Kapitel/Anhang}
\end{appendix}

% >>> Literaturverzeichnis <<<

\renewcommand{\bibname}{Quellenverzeichnis}
\addcontentsline{toc}{chapter}{\bibname}
\bibliographystyle{unsrt} 
\bibliography{BachelorArbeit}

\newpage
\thispagestyle{empty}
\ \\

% >>> Erklaerung <<<

\input{Kapitel/Erklaerung}

\end{document}