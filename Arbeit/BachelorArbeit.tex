\documentclass[11pt,a4paper,twoside]{report}

% Deutsche Spracheinstellungen
\usepackage[ngerman,german]{babel, varioref}
\usepackage[T1]{fontenc}
\usepackage[utf8]{inputenc}

%\usepackage{marvosym}

\usepackage{amsfonts}
\usepackage{amssymb}
\usepackage{amsmath}
\usepackage{amscd}
\usepackage{amstext}
\usepackage{float}
\usepackage{caption}
\usepackage{wrapfig}
\usepackage{setspace}
%\usepackage[onehalfspacing]{setspace}
\usepackage{threeparttable}
\usepackage{footnote}
\usepackage{feynmf}
\usepackage{bbm}
\usepackage{slashed}
\usepackage{textcomp}
\usepackage{multirow}

\newfloat{formel}{htbp}{for}
\floatname{formel}{Formel}

\onehalfspacing
%\setstretch {1.433}

\usepackage{longtable}

%\usepackage{bibgerm}

\usepackage{footnpag}

\usepackage{ifthen}                 %%% package for conditionals in TeX
\usepackage[amssymb]{SIunits}
%Fr textumflossene Bilder und Tablellen
%\usepackage{floatflt} - veraltet

%Fr Testzwecke aktivieren, zeigt labels und refs im Text an.
%\usepackage{showkeys}

% Abstand zwischen zwei Abs�zen nach DIN (1,5 Zeilen)
% \setlength{\parskip}{1.5ex}

% Einrckung am Anfang eines neuen Absatzes nach DIN (keine)
%\setlength{\parindent}{0pt}

% R�der definieren
% \setlength{\oddsidemargin}{0.3cm}
% \setlength{\textwidth}{15.6cm}

% bessere Bildunterschriften
%\usepackage[center]{caption2}


% Probleml�ungen beim Umgang mit Gleitumgebungen
\usepackage{float}

% Nummeriert bis zur Strukturstufe 3 (also <section>, <subsection> und <subsubsection>)
%\setcounter{secnumdepth}{3}

% Fhrt das Inhaltsverzeichnis bis zur Strukturstufe 3
%\setcounter{tocdepth}{3}

\usepackage{exscale}

\newenvironment{dsm} {\begin{displaymath}} {\end{displaymath}}
\newenvironment{vars} {\begin{center}\scriptsize} {\normalsize \end{center}}


\newcommand {\en} {\varepsilon_0}               % Epsilon-Null aus der Elektrodynamik
\newcommand {\lap} {\; \mathbf{\Delta}}         % Laplace-Operator
\newcommand {\R} { \mathbb{R} }                 % Menge der reellen Zahlen
\newcommand {\e} { \ \mathbf{e} }               % Eulersche Zahl
\renewcommand {\i} { \mathbf{i} }               % komplexe Zahl i
\newcommand {\N} { \mathbb{N} }                 % Menge der nat. Zahlen
\newcommand {\C} { \mathbb{C} }                 % Menge der kompl. Zahlen
\newcommand {\Z} { \mathbb{Z} }                 % Menge der kompl. Zahlen
\newcommand {\limi}[1]{\lim_{#1 \rightarrow \infty}} % Limes unendlich
\newcommand {\sumi}[1]{\sum_{#1=0}^\infty}
\newcommand {\rot} {\; \mathrm{rot} \,}         % Rotation
\newcommand {\grad} {\; \mathrm{grad} \,}       % Gradient
\newcommand {\dive} {\; \mathrm{div} \,}        % Divergenz
\newcommand {\dx} {\; \mathrm{d} }              % Differential d
\newcommand {\cotanh} {\; \mathrm{cotanh} \,}   %Cotangenshyperbolicus
\newcommand {\asinh} {\; \mathrm{areasinh} \,}  %Area-Sinus-Hyp.
\newcommand {\acosh} {\; \mathrm{areacosh} \,}  %Area-Cosinus-H.
\newcommand {\atanh} {\; \mathrm{areatanh} \,}  %Area Tangens-H.
\newcommand {\acoth} {\; \mathrm{areacoth} \,}  % Area-cotangens
\newcommand {\Sp} {\; \mathrm{Sp} \,}
\newcommand {\mbe} {\stackrel{\text{!}}{=}}     %Must Be Equal
\newcommand{\qed} { \hfill $\square$\\}
\newcommand{\midtilde}{\raisebox{-0,25\baselineskip}{\textasciitilde}}
\renewcommand{\i} {\imath}
\def\captionsngerman{\def\figurename{\textbf{Abb.}}}

%%%%%%%%%%%%%%%%%%%%%%%%%%%%%%%%%%%%%%%%%%%%%%%%%%%%%%%%%%%%%%%%%%%%%%%%%%%%
% SWITCH FOR PDFLATEX or LATEX
%%%%%%%%%%%%%%%%%%%%%%%%%%%%%%%%%%%%%%%%%%%%%%%%%%%%%%%%%%%%%%%%%%%%%%%%%%%%
%%%
\ifx\pdfoutput\undefined %%%%%%%%%%%%%%%%%%%%%%%%%%%%%%%%%%%%%%%%% LATEX %%%
%%%
\usepackage[dvips]{graphicx}       %%% graphics for dvips
\DeclareGraphicsExtensions{.eps,.ps}   %%% standard extension for included graphics
\usepackage[ps2pdf]{thumbpdf}      %%% thumbnails for ps2pdf
\usepackage[ps2pdf,                %%% hyper-references for ps2pdf
bookmarks=true,%                   %%% generate bookmarks ...
bookmarksnumbered=true,%           %%% ... with numbers
hypertexnames=false,%              %%% needed for correct links to figures !!!
breaklinks=true,%                  %%% breaks lines, but links are very small
linkbordercolor={0 0 1},%          %%% blue frames around links
pdfborder={0 0 112.0}]{hyperref}%  %%% border-width of frames
%                                      will be multiplied with 0.009 by ps2pdf
%
\hypersetup{ pdfauthor   = {Hannes Franke; Julius Tilly},
pdftitle    = {V301 Innenwiderstand und Leistungsanpassung}, pdfsubject  = {Protokoll FP}, pdfkeywords = {V301, Innenwiderstand, Leistungsanpassung},
pdfcreator  = {LaTeX with hyperref package}, pdfproducer = {dvips
+ ps2pdf} }
%%%
\else %%%%%%%%%%%%%%%%%%%%%%%%%%%%%%%%%%%%%%%%%%%%%%%%%%%%%%%%%% PDFLATEX %%%
%%%
\usepackage[pdftex]{graphicx}      %%% graphics for pdfLaTeX
\DeclareGraphicsExtensions{.pdf}   %%% standard extension for included graphics
\usepackage[pdftex]{thumbpdf}      %%% thumbnails for pdflatex
\usepackage[pdftex,                %%% hyper-references for pdflatex
bookmarks=true,%                   %%% generate bookmarks ...
bookmarksnumbered=true,%           %%% ... with numbers
hypertexnames=false,%              %%% needed for correct links to figures !!!
breaklinks=true,%                  %%% break links if exceeding a single line
linkbordercolor={0 0 1},
linktocpage]{hyperref} %%% blue frames around links
%                                  %%% pdfborder={0 0 1} is the default
\hypersetup{
pdftitle    = {V301 Innenwiderstand und Leistungsanpassung}, 
pdfsubject  = {Protokoll AP}, 
pdfkeywords = {V301, Innenwiderstand, Leistungsanpassung},
pdfsubject  = {Protokoll AP},
pdfkeywords = {V301, Innenwiderstand, Leistungsanpassung}}
%                                  %%% pdfcreator, pdfproducer,
%                                      and CreationDate are automatically set
%                                      by pdflatex !!!
\pdfadjustspacing=1                %%% force LaTeX-like character spacing
\usepackage{epstopdf}
%
\fi %%%%%%%%%%%%%%%%%%%%%%%%%%%%%%%%%%%%%%%%%%%%%%%%%%% END OF CONDITION %%%
%%%%%%%%%%%%%%%%%%%%%%%%%%%%%%%%%%%%%%%%%%%%%%%%%%%%%%%%%%%%%%%%%%%%%%%%%%%%
% seitliche Tabellen und Abbildungen
%\usepackage{rotating}
\usepackage{ae}
\usepackage{
  array,
  booktabs,
  dcolumn
}
\makeatletter 
  \renewenvironment{figure}[1][] {% 
    \ifthenelse{\equal{#1}{}}{% 
      \@float{figure} 
    }{% 
      \@float{figure}[#1]% 
    }% 
    \centering 
  }{% 
    \end@float 
  } 
  \makeatother 


  \makeatletter 
  \renewenvironment{table}[1][] {% 
    \ifthenelse{\equal{#1}{}}{% 
      \@float{table} 
    }{% 
      \@float{table}[#1]% 
    }% 
    \centering 
  }{% 
    \end@float 
  } 
  \makeatother 
%\usepackage{listings}
%\lstloadlanguages{[Visual]Basic}
%\allowdisplaybreaks[1]
%\usepackage{hycap}
%\usepackage{fancyunits}

\usepackage{german}
\usepackage{graphicx}
\usepackage[latin1]{inputenc}
\usepackage{nicehead.sty}
\usepackage{epsfig}
\usepackage{amssymb}
\usepackage{amsmath}
\usepackage{tabularx}
\usepackage{calc}
\usepackage[vflt]{floatflt}
\usepackage{units}
\usepackage{upgreek}
\usepackage[pdfborder={0 0 0}, hypertexnames=false]{hyperref}
\usepackage[official]{eurosym}

% Seitenstil
\pagestyle{emheadings}

% Breite des Textblocks und der Raender
\setlength{\evensidemargin}{0mm}
\setlength{\oddsidemargin}{13mm}
\setlength{\textwidth}{145mm}

\setcounter{secnumdepth}{3}   % Tiefe der Kapitelnummerierung
\setcounter{tocdepth}{3}      % Tiefe der Kapitelnummerierung im Inhaltsverzeichnis

\setcounter{totalnumber}{3}
\renewcommand{\floatpagefraction}{0.99}


\mathcode`\,="013B
 \setlength{\parskip}{1.5ex}
\begin{document}
\pagenumbering{Roman}

% Anmerkung: Die Seitenraender wurden asymmetrisch gewaehlt,
%            damit genug Platz fuer eine Klemmbindung da ist.
%            Da neue Kapitel auf der rechten Seite (ungerade
%            Seitennummer) beginnen sollten, muss ggf. am Ende
%            des vorhergehenden Kapitels eine Leerseite
%            eingefuegt werden:
%
%            \newpage
%            \thispagestyle{empty}
%            \ \\
%            \newpage
%
%            Die Seitenraender koennen aber auch in der Datei Tex/global.tex
%            veraendert werden.

% >>> Titelseite <<<

\newcommand{\thetitle}{Formfaktoren des semileptonischen $D \rightarrow K l \bar \nu$ Zerfalls}

\thispagestyle{empty}
\begin{center}
\Huge\textbf{\thetitle}
\vfill
% Note that the size is given in normal parentheses
% instead of curly brackets.
% Define external vertices from bottom to top
\vfill
\Large
Bachelorarbeit \\ zur Erlangung des akademischen Grades \\ Bachelor of Science \\
\vspace{20pt}
\normalsize
vorgelegt von \\[5pt]
{\Large Dimitrios Skodras} \\[5pt]
geboren in Aschaffenburg \\
\vspace{20pt}
Lehrstuhl für Theoretische Physik IV \\ Fakultät Physik \\
Technische Universität Dortmund \\ 2014
\end{center}
\newpage

% >>> Gutachterseite <<<

\thispagestyle{empty}
\vspace*{\fill}
\begin{tabbing}
1. Gutachter : \=\kill
1. Gutachter : \>Prof. Dr. Musterfrau \\[11pt]
2. Gutachter : \>Prof. Dr. Mustermann \\[11pt]
\end{tabbing}
\vspace{11pt}
Datum des Einreichens der Arbeit: TT.\,Monat\,JJJJ
\newpage

% >>> Kurzfassung/Abstract <<<

\thispagestyle{empty}
%Kuzfassung
\section*{Kurzfassung}
Hier folgt eine kurze Zusammenfassung des Inhalts und der Ergebnisse der Arbeit in deutscher Sprache.\\
\ \\

\newpage

% >>> Hauptteil <<<

\addcontentsline{toc}{chapter}{Inhaltsverzeichnis}
\tableofcontents\newpage
\addcontentsline{toc}{chapter}{Abbildungsverzeichnis}
\listoffigures\newpage
\addcontentsline{toc}{chapter}{Tabellenverzeichnis}
\listoftables\newpage

\setcounter{page}{0}
\pagenumbering{arabic}

\chapter{Einleitung}


\chapter{Theorie}
Ehe die Matrixelemente errechnet werden, ist es erforderlich, Kenntnis von beteiligten Zusammenhängen zu haben. Zum einen werden die beim Zerfall beteiligten
Teilchen und ihre fundamentalen Wechselwirkungen beleuchtet. Dazu gehören die D- und K-Mesonen mit Quarks als Konstituenten, sowie den Leptonen als Teilchen
und die schwache Wechselwirkung 
Da es sich um sehr kleine Teilchen handelt ist eine Betrachtung der Quantenmechanik erforderlich. 
Da sie zusätzlich leicht und daher hinsichtlich der Lichtgeschwindigkeit schnell sind, bleiben Gesetzmäßigkeiten der relativistische Kinematik nicht aus.
Zum anderen werden allgemein Formfaktorparametrisierungen und im speziellen die Reihenentwicklung thematisiert.
\section{Voraussetzungen moderner Physik}
Die physikalischen Errungenschaften aus der ersten Hälfte des 20. Jahrhunderts stellen aufgrund ihrer Richtigkeit und Exaktheit die Voraussetzungen an moderne 
Theorien, dass diese immer gelten müssen. Hiermit sind die spezielle Relativitätstheorie und die Quantenmechanik gemeint, sowie ihre Vereinigung, die relativistische
Quantenmechanik gemeint. Davon sind im Rahmen dieser Arbeit die relativistische Kinematik, die Dirac-Gleichung sowie die störungstheoretische ``Goldene Regel'' von Fermi
von Bedeutung.

\subsection{Relativistische Kinematik}
Die SRT stellt die fundamentale Forderung, dass die Form der Naturgesetze unabhängig vom Inertialsystem gleich ist. Mit der Lichtgeschwindigkeit $c$ als
größte vorkommende Geschwindigkeit ist sie ebenfalls in allen Inertialsystemen gleich groß. Die relativistische Energie-Impuls Beziehung 
$E^2 = \left(mc^2\right)^2 + \left(\vec pc\right)^2$ beschreibt einen allgemeinen Zusammenhang zwischen der Energie $E$, der Masse $m$ und dem Impuls $\vec p$.
In der Hochenergiephysik ($E_{\text{CMS}} \approx 10^4$ GeV) gilt der hochrelativistische Grenzfall, bei dem Energie hauptsächlich durch den Impuls bestimmt wird.
In natürlichen Einheiten wird $c = 1$ gesetzt, was zu $E^2 = \left|p\right|^2$ führt.

Zur Beschreibung der Bewegung von relativistischen Teilchen wird wegen der Energie-Impuls-Beziehung und der Verknüpfung von Raum und Zeit ($x = t$) das Konzept
der Vierer-Vektoren eingeführt. Es gestaltet sich so, dass die Zeit als 0. Komponente des Raums und die Energie als 0. Komponente des Impulses angesehen werden
kann, was die 4-Dimensionalität zeigt \cite{RelKin}.
\begin{align}
 x^\mu &= (t,\,x,\,y,\,z)^\mu \hspace{2cm} \text{Vierer-Ort}\\
 p^\mu &= (E,\,p_x,\,p_y,\,p_z)^\mu \hspace{1.35cm} \text{Vierer-Impuls}
\end{align}
Der Index $\mu$ kann ganzzahlige Werte zwischen 0 und 4 annehmen und steht für die jeweilige Komponente des Vektors. Im Gegensatz zu euklidischen Räumen 
kann ein Skalarprodukt zweier Vierer-Vektoren nur dann beschrieben werden, wenn einer kovariant (Index unten) und der andere kontravariant (Index oben) ist.
Diese Überführung geschieht durch die Minkowskimetrik, die die Norm unter Lorentz-Transformationen konstant lässt
\begin{equation}
  p^2 = p^\mu \eta_{\mu \nu} p^\nu = p^\mu p_\mu = E^2 - \vec p^2 = m^2,
\end{equation}
was wieder die relativistische Energie-Impuls-Beziehung ist. Die Einsteinsche Summenkonvention wird hierbei angewandt.

\subsection{Dirac-Gleichung}
Aus der nicht-relativistischen Schrödinger-Gleichung als quantenmechanische Wellengleichung ergibt sich durch erste Quantisierung eine Ersetzung von Energie und
Impuls durch partielle Differentialoperatoren
\begin{align*}
 E \rightarrow \text{i}\hbar\partial_t, \quad p \rightarrow -\text{i}\hbar\nabla.
\end{align*}
Da die Schrödinger-Gleichung nicht lorentzinvariant ist, sind andere Ansätze durchgeführt worden, die der zuvor genannten Energie-Impuls-Beziehung $p^\mu p_\mu = m^2$
genügen. Die Klein-Gordon-Gleichung für spinlose Teilchen, die aus ihr direkt folgt, ist zwar relativistisch korrekt, weist jedoch keine positiv definite 
Wahrscheinlichkeitsdichte auf. Die Dirac-Gleichung für Spin$\frac12$-Teilchen ist ebenfalls unter Lorentztransformationen invariant und besitzt nun
zusätzlich eine positiv definite Wahrscheinlichkeitsdichte, was eine Interpretation ihrer Lösungen als Wahrscheinlichkeitsamplitude zulässt. Mit $\hbar = 1$
lautet die Dirac-Gleichung
\begin{equation}
 (\slashed{p} - m )\psi(x) = 0,
\end{equation}
mit $\psi(x)$ als Wellenfunktion und $\slashed{p}$ als Impulsoperator in der Feynman-Slash-Notation
\begin{equation}
 \slashed{A} := \gamma^\mu A_\mu.
\end{equation}



\subsection{Fermis Goldene Regel}

\section{Stadardmodell der Elementarteilchenphysik}
Das SM setzt sich aus zwei definierenden Eigenschaften zusammen, den Teilchen und den Eichsymmetrien, die diese beschreiben. Die sichtbare Materie wird 
in drei Generationen von fundamentalen Fermionen zusammengesetzt, deren Attribute in Tabelle \ref{tab_particlezoo} dargestellt sind. 

\begin{table}[H]
\begin{tabular}{c|cccc|ccc} \toprule \toprule
 Generation & & $m$ in MeV & $\tau$ in s & $q$ in e & & $m$ in MeV & $q$ in e\\
 \midrule
  1 & e & 0,511 & stabil & -1 & u & 1 & +$\frac23$\\
\\\bottomrule
 \end{tabular}
\caption{elementarer Teilchenzoo \cite{SM}}
\label{tab_particlezoo}
\end{table}


\subsection{Teilcheninhalt}
elementare Teilchen
d-meson, k-meson
\subsection{Schwache Wechselwirkung}
parität
V-A-Theorie
vierstromwechselwirkung
ckm-matrix


\section{Parametrisierung von Formfaktoren}
parametrisierungen nennen und auf den speziellen näher eingehen
AP1 + BP2
z-expansion

\cite{b:feynman}
\chapter{Messungen}
\section{Energiebereich von $q^2$}
\section{Ermittlung der Formfaktoren}


\chapter{Zusammenfassung und Ausblick}

Hier sollen die Ergebnisse zusammengefasst und weiterf\"uhrende Untersuchungen diskutiert werden. 


% >>> Anhang <<<

\begin{appendix}
%\input{Kapitel/Anhang}
\end{appendix}

% >>> Literaturverzeichnis <<<

\renewcommand{\bibname}{Quellenverzeichnis}
\addcontentsline{toc}{chapter}{\bibname}
%\bibliographystyle{unsrt} 
%\bibliography{BachelorArbeit}
\begin{thebibliography}{xxx}
 \bibitem[1]{RelKin}M. zur Nedden ``Detektoren der Elementarteilchenphysik'' \href{http://www-hera-b.desy.de/people/nedden/lectures/05_06/dettph/dettph_cont.pdf}{http://www-hera-b.desy.de/people/nedden/lectures/05\_06/dettph/dettph\_cont.pdf}, 2006
 \bibitem[2]{DissForm}N. Offen ``B-Zerfallsformfaktoren aus QCD-Summenregeln''\\ \href{http://d-nb.info/987811061}{http://d-nb.info/987811061}, 2008
 \bibitem[3]{SM} W.N. Cottingham, D.A. Greenwood, ``An Introduction to the Standard Model of Particle Physics'' ISBN-13 978-0-511-27377-3, 2nd Edition, 2007
 \bibitem[30]{bla}{Versuch V28 Elektronen-Spin-Resonanz }
\end{thebibliography}


\newpage
\thispagestyle{empty}
\ \\

% >>> Erklaerung <<<

\thispagestyle{empty}
\begin{center}
\section*{Eidesstattliche Versicherung}
\end{center}
\vspace*{1cm}
\noindent
Ich versichere hiermit an Eides statt, dass ich die vorliegende Bachelorarbeit mit dem Titel ''{\thetitle}'' selbst\"andig und ohne unzul\"assige fremde Hilfe erbracht habe. Ich habe keine anderen als die angegebenen
Quellen und Hilfsmittel benutzt sowie w\"ortliche und sinngem\"a\ss e Zitate kenntlich gemacht.
Die Arbeit hat in gleicher oder \"ahnlicher Form noch keiner Pr\"ufungsbeh\"orde vorgelegen.
\vspace*{1cm}
\ \\
\ \\
\line(1,0){150} \hfill \line(1,0){150}\\
Ort, Datum \hfill Unterschrift \hspace*{3cm}
\vspace*{1.5cm}

\subsection*{Belehrung}
Wer vors\"atzlich gegen eine die T\"auschung \"uber Pr\"ufungsleistungen betreffende Regelung einer Hochschulpr\"ufungsordnung
verst\"o\ss t handelt ordnungswidrig. Die Ordnungswidrigkeit kann mit einer Geldbu\ss e von bis zu \unit[50.000,00]{\euro} geahndet werden. Zust\"andige Verwaltungsbeh\"orde f\"ur die Verfolgung und Ahndung von Ordnungswidrigkeiten ist
der Kanzler/die Kanzlerin der Technischen Universit\"at Dortmund. Im Falle eines mehrfachen oder sonstigen schwerwiegenden T\"auschungsversuches kann der Pr\"ufling zudem exmatrikuliert werden (\S\ 63 Abs. 5 Hochschulgesetz - HG - ).\\
\ \\
Die Abgabe einer falschen Versicherung an Eides statt wird mit Freiheitsstrafe bis zu 3 Jahren oder mit Geldstrafe bestraft.\\
\ \\
Die Technische Universit\"at Dortmund wird ggf. elektronische Vergleichswerkzeuge (wie z.B. die Software ''turnitin'') zur \"Uberpr\"ufung von Ordnungswidrigkeiten in Pr\"ufungsverfahren nutzen.\\
\ \\
Die oben stehende Belehrung habe ich zur Kenntnis genommen.
\vspace*{1cm}
\ \\
\ \\
\line(1,0){150} \hfill \line(1,0){150}\\
Ort, Datum \hfill Unterschrift \hspace*{3cm}
\vspace*{\fill}

\end{document}