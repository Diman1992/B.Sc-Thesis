\documentclass[11pt,a4paper,twoside]{report}
\input{Tex/longheader.tex}
\input{Tex/global}
 \setlength{\parskip}{1.5ex}
\begin{document}
\pagenumbering{Roman}

% Anmerkung: Die Seitenraender wurden asymmetrisch gewaehlt,
%            damit genug Platz fuer eine Klemmbindung da ist.
%            Da neue Kapitel auf der rechten Seite (ungerade
%            Seitennummer) beginnen sollten, muss ggf. am Ende
%            des vorhergehenden Kapitels eine Leerseite
%            eingefuegt werden:
%
%            \newpage
%            \thispagestyle{empty}
%            \ \\
%            \newpage
%
%            Die Seitenraender koennen aber auch in der Datei Tex/global.tex
%            veraendert werden.

% >>> Titelseite <<<

\newcommand{\thetitle}{Formfaktoren des semileptonischen $D \rightarrow K l \bar \nu$ Zerfalls}

\thispagestyle{empty}
\begin{center}
\Huge\textbf{\thetitle}
\vfill
% Note that the size is given in normal parentheses
% instead of curly brackets.
% Define external vertices from bottom to top
\vfill
\Large
Bachelorarbeit \\ zur Erlangung des akademischen Grades \\ Bachelor of Science \\
\vspace{20pt}
\normalsize
vorgelegt von \\[5pt]
{\Large Dimitrios Skodras} \\[5pt]
geboren in Aschaffenburg \\
\vspace{20pt}
Lehrstuhl für Theoretische Physik IV \\ Fakultät Physik \\
Technische Universität Dortmund \\ 2014
\end{center}
\newpage

% >>> Gutachterseite <<<

\thispagestyle{empty}
\vspace*{\fill}
\begin{tabbing}
1. Gutachter : \=\kill
1. Gutachter : \>Prof. Dr. Musterfrau \\[11pt]
2. Gutachter : \>Prof. Dr. Mustermann \\[11pt]
\end{tabbing}
\vspace{11pt}
Datum des Einreichens der Arbeit: TT.\,Monat\,JJJJ
\newpage

% >>> Kurzfassung/Abstract <<<

\thispagestyle{empty}
%Kuzfassung
\section*{Kurzfassung}
Hier folgt eine kurze Zusammenfassung des Inhalts und der Ergebnisse der Arbeit in deutscher Sprache.\\
\ \\

\newpage

% >>> Hauptteil <<<

\addcontentsline{toc}{chapter}{Inhaltsverzeichnis}
\tableofcontents\newpage
\addcontentsline{toc}{chapter}{Abbildungsverzeichnis}
\listoffigures\newpage
\addcontentsline{toc}{chapter}{Tabellenverzeichnis}
\listoftables\newpage

\setcounter{page}{0}
\pagenumbering{arabic}

\chapter{Einleitung}


\chapter{Theorie}
Ehe die Matrixelemente errechnet werden, ist es erforderlich, Kenntnis von beteiligten Zusammenhängen zu haben. Zum einen werden die beim Zerfall beteiligten
Teilchen und ihre fundamentalen Wechselwirkungen beleuchtet. Dazu gehören die D- und K-Mesonen mit Quarks als Konstituenten, sowie den Leptonen als Teilchen
und die schwache Wechselwirkung 
Da es sich um sehr kleine Teilchen handelt ist eine Betrachtung der Quantenmechanik erforderlich. 
Da sie zusätzlich leicht und daher hinsichtlich der Lichtgeschwindigkeit schnell sind, bleiben Gesetzmäßigkeiten der relativistische Kinematik nicht aus.
Zum anderen werden allgemein Formfaktorparametrisierungen und im speziellen die Reihenentwicklung thematisiert.
\section{Voraussetzungen moderner Physik}
Die physikalischen Errungenschaften aus der ersten Hälfte des 20. Jahrhunderts stellen aufgrund ihrer Richtigkeit und Exaktheit die Voraussetzungen an moderne 
Theorien, dass diese immer gelten müssen. Hiermit sind die spezielle Relativitätstheorie und die Quantenmechanik gemeint, sowie ihre Vereinigung, die relativistische
Quantenmechanik gemeint. Davon sind im Rahmen dieser Arbeit die relativistische Kinematik, die Dirac-Gleichung sowie die störungstheoretische ``Goldene Regel'' von Fermi
von Bedeutung.

\subsection{Relativistische Kinematik}
Die SRT stellt die fundamentale Forderung, dass die Form der Naturgesetze unabhängig vom Inertialsystem gleich ist. Mit der Lichtgeschwindigkeit $c$ als
größte vorkommende Geschwindigkeit ist sie ebenfalls in allen Inertialsystemen gleich groß. Die relativistische Energie-Impuls Beziehung 
$E^2 = \left(mc^2\right)^2 + \left(\vec pc\right)^2$ beschreibt einen allgemeinen Zusammenhang zwischen der Energie $E$, der Masse $m$ und dem Impuls $\vec p$.
In der Hochenergiephysik ($E_{\text{CMS}} \approx 10^4$ GeV) gilt der hochrelativistische Grenzfall, bei dem Energie hauptsächlich durch den Impuls bestimmt wird.
In natürlichen Einheiten wird $c = 1$ gesetzt, was zu $E^2 = \left|p\right|^2$ führt.

Zur Beschreibung der Bewegung von relativistischen Teilchen wird wegen der Energie-Impuls-Beziehung und der Verknüpfung von Raum und Zeit ($x = t$) das Konzept
der Vierer-Vektoren eingeführt. Es gestaltet sich so, dass die Zeit als 0. Komponente des Raums und die Energie als 0. Komponente des Impulses angesehen werden
kann, was die 4-Dimensionalität zeigt \cite{RelKin}.
\begin{align}
 x^\mu &= (t,\,x,\,y,\,z)^\mu \hspace{2cm} \text{Vierer-Ort}\\
 p^\mu &= (E,\,p_x,\,p_y,\,p_z)^\mu \hspace{1.35cm} \text{Vierer-Impuls}
\end{align}
Der Index $\mu$ kann ganzzahlige Werte zwischen 0 und 4 annehmen und steht für die jeweilige Komponente des Vektors. Im Gegensatz zu euklidischen Räumen 
kann ein Skalarprodukt zweier Vierer-Vektoren nur dann beschrieben werden, wenn einer kovariant (Index unten) und der andere kontravariant (Index oben) ist.
Diese Überführung geschieht durch die Minkowskimetrik, die die Norm unter Lorentz-Transformationen konstant lässt
\begin{equation}
  p^2 = p^\mu \eta_{\mu \nu} p^\nu = p^\mu p_\mu = E^2 - \vec p^2 = m^2,
\end{equation}
was wieder die relativistische Energie-Impuls-Beziehung ist. Die Einsteinsche Summenkonvention wird hierbei angewandt.

\subsection{Dirac-Gleichung}
Aus der nicht-relativistischen Schrödinger-Gleichung als quantenmechanische Wellengleichung ergibt sich durch erste Quantisierung \cite{TeilFortgeschr} eine Ersetzung von Energie 
und Impuls durch partielle Differentialoperatoren
\begin{align*}
 E \rightarrow \text{i}\hbar\partial_t, \quad p \rightarrow -\text{i}\hbar\nabla.
\end{align*}
Da die Schrödinger-Gleichung nicht lorentzinvariant ist, sind andere Ansätze durchgeführt worden, die der zuvor genannten Energie-Impuls-Beziehung $p^\mu p_\mu = m^2$
genügen. Die Klein-Gordon-Gleichung für spinlose Teilchen, die aus ihr direkt folgt, ist zwar relativistisch korrekt, weist jedoch keine positiv definite 
Wahrscheinlichkeitsdichte auf. Die Dirac-Gleichung für Spin$\frac12$-Teilchen ist ebenfalls unter Lorentztransformationen invariant und besitzt nun
zusätzlich eine positiv definite Wahrscheinlichkeitsdichte, was eine Interpretation ihrer Lösungen als Wahrscheinlichkeitsamplitude zulässt. Daher muss diese
Gleichung linear in der ersten Zeit- und Ortsableitung sein und der folgenden Schrödinger-Form genügen \cite{RelQuantMech}
\begin{equation}
 \text{i} \partial_t \psi = \left(-\text{i}\alpha^k\partial_k + \beta m\right)\psi = H \psi,
 \label{eq_diracSchroedinger}
\end{equation}
mit $\hbar = 1$ und $H$ dem diracschen Hamiltonian. $\alpha^k = \gamma^0\gamma^k$ und $\beta = \gamma^0$ sind die historischen Dirac-Matrizen. Gelöst wird
diese Schrödinger-Form von der folgenden Dirac-Gleichung
\begin{equation}
 (\slashed{p} - m )\psi = 0,
 \label{eq_dirac}
\end{equation}
mit $\slashed{p}$ als Impulsoperator in der Feynman-Slash-Notation
\begin{equation}
 \slashed{A} := \gamma^\mu A_\mu = \gamma^0 A_0 - \gamma^i\cdot A_i
\end{equation}
und $\psi$ als Wellenfunktion, da $\gamma^\mu \in \text{M}^{4\text{x}4}$, mit den vier Freiheitsgraden: Teilchen, Antiteilchen, Spin-Up, Spin-Down.
Die $\gamma$-Matrizen in der Dirac-Pauli-Notation lauten
\begin{equation*}
  \gamma^0 = \begin{pmatrix} 
              \mathbbm{1} & 0 \\
              0 & -\mathbbm{1}
             \end{pmatrix},\quad \gamma^i = \begin{pmatrix}
					0 & \sigma^i\\
					-\sigma^i & 0
				      \end{pmatrix} \quad \text{und}\quad \gamma^5 = \begin{pmatrix}
									  0 & \mathbbm{1}\\
									  \mathbbm{1} & 0
									   \end{pmatrix}
\end{equation*}
mit $\mathbbm{1}$ als 2x2-Einheitsmatrix und den $\sigma^i$ als Paulimatrizen
\begin{equation*}
 \sigma^1 = \begin{pmatrix}
             0 & 1\\
             1 & 0
            \end{pmatrix},\quad \sigma^2 = \begin{pmatrix}
					    0 & -\text{i}\\
					    \text{i} & 0
					    \end{pmatrix}\quad \text{und} \quad\sigma^3 = \begin{pmatrix}
									    1 & 0\\
									    0 & -1
									    \end{pmatrix}.									    
\end{equation*}
Die Lösungen der freien Dirac-Gleichung sind folgende Wellenfunktionen
\begin{align}
 \psi_+(x) = u(p)^{(1,2)} \e^{-ip^\mu x_mu}\qquad\text{und}\qquad \psi_-(x) = v(p)^{(1,2)} \e^{+ip^\mu x_mu},
 \label{eq_diracLsg}
\end{align}
die eingesetzt in \eqref{eq_dirac} die Gleichungen für die Spinoren
\begin{align}
 (\slashed{p} - m )u^{(1,2)} &= \bar u^{(1,2)}(\slashed{p}-m) = 0 \\
 (\slashed{p} + m )v^{(1,2)} &= \bar v^{(1,2)}(\slashed{p}+m) = 0,
\end{align}
ergeben, wobei $\slashed{p}$ hier nun kein Operator, sondern der Eigenwert des Viererimpulses der ebenen Welle \eqref{eq_diracLsg} ist und $\bar a = a^\dagger \gamma^0$,
mit $\dagger$ als komplexe Konjugation und Transposition. Lösungen dieser Bispinoren der Teilchen ($u$) und Antiteilchen ($v$), die nur von Impuls und Spin abhängen,
lauten nun
\begin{equation}
 u(p,s)^{(1,2)} = \mathcal{N} \begin{pmatrix}
                          \chi^{(1,2)}\\
                          \frac{\vec \sigma \vec p}{E+m} \chi_s^{(1,2)}
                         \end{pmatrix} \qquad \text{und}\qquad v(p,s)^{(1,2)} = \mathcal{N} \begin{pmatrix}
                          \frac{\vec \sigma \vec p}{E+m} \chi_s^{(2,1)}\\
                          \chi^{(2,1)}                          
                         \end{pmatrix},
\end{equation}
mit einer Normierung $\mathcal{N}$, die für gewöhnlich $\sqrt{E + m}$ gewählt wird und dem nicht-relativistischen Spinor für Spin $\frac12$-Teilchen $\chi_s$.
Die Energieeigenwerte der freien Lösungen für Teilchen $\psi_+$ sind positiv und die der Antiteilchen $\psi_-$ negativ, was mit der Löchertheorie interpretiert
wird. Wie bereits erwähnt, existiert bei der Dirac-Gleichung ein Wahrscheinlichkeitsstrom $j^\mu$, der der lorentzinvarianten Kontinuitätsgleichung
\begin{equation}
 \partial_\mu j^\mu = 0
\end{equation}
genügt. Zur Ermittlung wird \eqref{eq_diracSchroedinger} von links mit der komplex konjugierten Wellenfunktion und die komplex konjugierte
Form von \eqref{eq_diracSchroedinger} von rechts mit der normalen Wellenfunktion multipliziert. Hierzu dient die Gleichheit 
$(\gamma^\mu)^\dagger = \gamma^0\gamma^\mu\gamma^0$. Die daraus resultierenden Gleichungen voneinander abgezogen ergeben die eben genannte Kontinuitätsgleichung.
Der Wahrscheinlichkeitsstrom lautet somit
\begin{equation}
 j^\mu = \bar \psi \gamma^\mu \psi.
\end{equation}


\subsection{Fermis Goldene Regel}

\section{Stadardmodell der Elementarteilchenphysik}
Das SM setzt sich aus zwei definierenden Eigenschaften zusammen, den Teilchen und den Eichsymmetrien, die diese beschreiben. Die sichtbare Materie wird 
in drei Generationen von fundamentalen Fermionen zusammengesetzt, deren Attribute in Tabelle \ref{tab_particlezoo} dargestellt sind. 

\begin{table}[H]
\begin{tabular}{c|cccc|ccc} \toprule \toprule
 Generation & & $m$ in MeV & $\tau$ in s & $q$ in e & & $m$ in MeV & $q$ in e\\
 \midrule
  1 & e & 0,511 & stabil & -1 & u & 1 & +$\frac23$\\
\\\bottomrule
 \end{tabular}
\caption{elementarer Teilchenzoo \cite{SM}}
\label{tab_particlezoo}
\end{table}


\subsection{Teilcheninhalt}
elementare Teilchen
d-meson, k-meson
\subsection{Schwache Wechselwirkung}
parität
V-A-Theorie
vierstromwechselwirkung
ckm-matrix


\section{Parametrisierung von Formfaktoren}
parametrisierungen nennen und auf den speziellen näher eingehen
AP1 + BP2
z-expansion

\cite{b:feynman}
\chapter{Messungen}
\section{Energiebereich von $q^2$}
\section{Ermittlung der Formfaktoren}


\chapter{Zusammenfassung und Ausblick}

Hier sollen die Ergebnisse zusammengefasst und weiterf\"uhrende Untersuchungen diskutiert werden. 


% >>> Anhang <<<

\begin{appendix}
%\input{Kapitel/Anhang}
\end{appendix}

% >>> Literaturverzeichnis <<<

\renewcommand{\bibname}{Quellenverzeichnis}
\addcontentsline{toc}{chapter}{\bibname}
%\bibliographystyle{unsrt} 
%\bibliography{BachelorArbeit}
\begin{thebibliography}{xxx}
 \bibitem[1]{RelKin}Nedden zur, M.: \textit{Detektoren der Elementarteilchenphysik}[pdf]\\ \href{http://www-hera-b.desy.de/people/nedden/lectures/05_06/dettph/dettph_cont.pdf}{http://www-hera-b.desy.de/people/nedden/lectures/05\_06/dettph/dettph\_cont.pdf}, 2006
 \bibitem[2]{TeilFortgeschr}Schleper, P.: \textit{Teilchenphysik für Fortgeschrittene}[pdf]\\ \href{http://www.desy.de/~schleper/lehre/}{http://www.desy.de/\midtilde schleper/lehre/}, 2011
 \bibitem[3]{RelQuantMech}Bjorken, J.D., Drell, S.D.: \textit{Relativstic Quantum Mechanics}, 1964, ISBN-13 978-0072320022 
 \bibitem[4]{SM} Cottingham, W.N., Greenwood, D.A.: \textit{An Introduction to the Standard Model of Particle Physics} 2nd Edition, 2007, ISBN-13 978-0-511-27377-3 
 \bibitem[5]{DissForm}Offen, N.: \textit{B-Zerfallsformfaktoren aus QCD-Summenregeln}\\ \href{http://d-nb.info/987811061}{http://d-nb.info/987811061}, 2008
 \bibitem[30]{bla}{Versuch V28 Elektronen-Spin-Resonanz }
\end{thebibliography}


\newpage
\thispagestyle{empty}
\ \\

% >>> Erklaerung <<<

\input{Kapitel/Erklaerung}

\end{document}