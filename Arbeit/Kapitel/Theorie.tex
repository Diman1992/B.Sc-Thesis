\chapter{Theoretische Grundlagen}
\section{Dies und das}
\label{s:diesunddas}
Alles, was man an theoretischen Grundlagen f\"ur das Verst\"andnis braucht. Zum Beispiel Gleichung \ref{e:gleichung}.

\begin{center}
\begin{equation}
\label{e:gleichung}
r = \int_{0}^{r} dr' = \int_{t(0)}^{t(r)}\frac{dr'}{dt'}dt' = \int_{t(0)}^{t(r)}\frac{dr'}{dN}\frac{dN}{dt'}dt'
\end{equation} 
\end{center}
\ \\ % Leerzeile

\noindent % Verhindert die Einrueckung der ersten Zeile
Kleinere mathematische Ausdr\"ucke, wie z.B. $A^{2}+B^{2}=C^{2}$ oder $v_{i}=\frac{s_{i}}{t_{i}}$ k\"onnen auch direkt
im Text benutzt werden. Schlie\ss lich gibt es noch Aufz\"ahlungen:

\begin{itemize}
\item etwas Wichtiges
\item weniger wichtig
\item fast vernachl\"assigbar
\item unbedingt notwendig
\end{itemize}

\noindent
Oder auch mit Nummerierung:

\begin{enumerate}
\item etwas Wichtiges
\item weniger wichtig
\item fast vernachl\"assigbar
\item unbedingt notwendig
\end{enumerate}


\section{Anderes}
\label{s:anderes}
Hier steht etwas, dass noch nicht in Abschnitt \ref{s:diesunddas} behandelt wurde. Alle Messungen sind in Kapitel \ref{c:Messungen} ab Seite \pageref{c:Messungen} beschrieben. Weitere lustige Geschichten findet man in \cite{b:feynman}.