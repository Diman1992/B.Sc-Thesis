
\documentclass[hyperref={pdfpagelabels=false}]{beamer}
% Die Hyperref Option hyperref={pdfpagelabels=false} verhindert die Warnung:
% Package hyperref Warning: Option `pdfpagelabels' is turned off
% (hyperref)                because \thepage is undefined. 
% Hyperref stopped early 
%
% \input{longheader.tex}
% \input{global.tex}

\usepackage{lmodern}
% Das Paket lmodern erspart die folgenden Warnungen:
% LaTeX Font Warning: Font shape `OT1/cmss/m/n' in size <4> not available
% (Font)              size <5> substituted on input line 22.
% LaTeX Font Warning: Size substitutions with differences
% (Font)              up to 1.0pt have occurred.
%

% % % % % % % % % % % % % % % % % % % % % % % % % % % % % % % % % % % % % % % % % % % %
\usepackage{siunitx}
\sisetup{load-configurations=abbreviations}
\sisetup{
	%locale=DE,
	seperr=true,                    % Fehler anzeigen
	tightpm,                        % Abstand zwischen Fehler verringern
	tophrase={{\text{ bis }}},
	fraction=nice,
	per-mode=fraction,
	free-standing-units=true,
	space-before-unit=true,
	use-xspace=true,
	group-separator={{\text{~}}},
	list-final-separator={{\text{ und }}}
}
\usepackage{natbib}
\usepackage[labelformat=empty]{caption}
\usepackage{movie15}
\usepackage{xcolor,colortbl}
\usepackage{slashed}
\usepackage{amsfonts}
\usepackage{amssymb}
\usepackage{amsmath}
\usepackage{amscd}
\usepackage{amstext}
\usepackage[ngerman,german]{babel, varioref}
\usepackage[T1]{fontenc}
\usepackage[utf8]{inputenc}
\usepackage{xfrac}

% % % % % % % % % % % % % % % % % % % % % % % % % % % % % % % % % % % % % % % % % % % % % % % % %
% Wenn \titel{\ldots} \author{\ldots} erst nach \begin{document} kommen,
% kommt folgende Warnung:
% Package hyperref Warning: Option `pdfauthor' has already been used,
% (hyperref) ... 
% Daher steht es hier vor \begin{document}

\title[Formfaktoren von $D\rightarrow K l \nu$]{Formfaktoren des semileptonischen $D\rightarrow K l \nu$ Zerfalls}  
\institute{Lehrstuhl f\"ur Theoretische Physik IV\\
Technische Universit\"at Dortmund}
\author{Dimitrios Skodras} 
\date{03.09.2014} 

% zusaetzlich ist das usepackage{beamerthemeshadow} eingebunden 
\usepackage{beamerthemeshadow}


%  \beamersetuncovermixins{\opaqueness<1>{25}}{\opaqueness<2->{15}}
%  sorgt dafuer das die Elemente die erst noch (zukuenftig) kommen 
%  nur schwach angedeutet erscheinen 
\beamersetuncovermixins{\opaqueness<1>{25}}{\opaqueness<2->{15}}
% klappt auch bei Tabellen, wenn teTeX verwendet wird\ldots

\beamertemplatenavigationsymbolsempty

\begin{document}

\setbeamertemplate{footline}
{%
  \leavevmode%
 \begin{beamercolorbox}%
    [wd=.5\paperwidth,ht=2.5ex,dp=1.125ex,leftskip=.3cm,rightskip=.3cm]%
    {author in head/foot}%
    \usebeamerfont{author in head/foot}%
    \hfill\insertshortauthor
  \end{beamercolorbox}%
  \begin{beamercolorbox}%
    [wd=.5\paperwidth,ht=2.5ex,dp=1.125ex,leftskip=.3cm ,rightskip=.3cm]%
    {title in head/foot}%
    \usebeamerfont{title in head/foot}%
    \insertshorttitle\hfill\insertframenumber{}
  \end{beamercolorbox}%
}%

\setbeamertemplate{caption}{\raggedright\insertcaption\par}
\captionsetup[figure]{font=small,skip=0pt}
\begin{frame}
\titlepage
\end{frame} 

\begin{frame}
\frametitle{Gliederung}
\tableofcontents
\end{frame} 

\newcommand{\tmotiv}{Der Zerfall}
\section{\tmotiv}

\begin{frame}
 \frametitle{Standardmodell}
 \begin{minipage}[h]{0.48\textwidth}
  \begin{itemize} 
  \item Teilcheninhalt:
  \begin{itemize}
   \item Leptonen: $e^+$, $\mu^+$, $\nu$
   \item Quarks: $\bar u$, $\bar d$, $s$, $c$
   \item Bosonen: $W^+$, $g$
  \end{itemize}  
  \item Fundamentale Wechselwirkungen: 
  \begin{itemize}
   \item starke Wechselwirkung (QCD)
   \item elektroschwache Wechselwirkung (GSW-Theorie)
  \end{itemize}  
 \end{itemize}
 \end{minipage}
 \begin{minipage}[h]{0.48\textwidth}
 \begin{figure}[h]
  \includegraphics[width = 2\textwidth]{../Abbildungen/Teilchen.png}
 \end{figure}
 \end{minipage}
\end{frame}

\begin{frame}
 \frametitle{Feynmangraph}
  \setcounter{framenumber}{5}
\begin{minipage}[h]{0.38\textwidth}
 \begin{enumerate}
  \item ruhendes $D$-Meson
 \end{enumerate}
\end{minipage}
\begin{minipage}[h]{0.58\textwidth}
  \begin{figure}[h]
  \includegraphics[width = 1.9\textwidth]{../Abbildungen/DFeyn1.png}
   \caption{Feynmangraph des $D\rightarrow Kl\nu$ Zerfalls}
 \end{figure}
\end{minipage}
\end{frame}

\begin{frame}
 \frametitle{Feynmangraph}
\setcounter{framenumber}{5}
\begin{minipage}[h]{0.38\textwidth}
 \begin{enumerate}
  \item ruhendes $D$-Meson
  \item propagiert in $t$
 \end{enumerate}
\end{minipage}
\begin{minipage}[h]{0.58\textwidth}
  \begin{figure}[h]
  \includegraphics[width = 1.9\textwidth]{../Abbildungen/DFeyn2.png}
   \caption{Feynmangraph des $D\rightarrow Kl\nu$ Zerfalls}
 \end{figure}
\end{minipage}
\end{frame}

\begin{frame}
 \frametitle{Feynmangraph}
 \setcounter{framenumber}{5}
\begin{minipage}[h]{0.38\textwidth}
 \begin{enumerate}
  \item ruhendes $D$-Meson
  \item propagiert in $t$.
  \item $c$ wandelt unter Abstrahlung von $W^+$ in $s$ 
 \end{enumerate}
\end{minipage}
\begin{minipage}[h]{0.58\textwidth}
  \begin{figure}[h]
  \includegraphics[width = 1.9\textwidth]{../Abbildungen/DFeyn3.png}
   \caption{Feynmangraph des $D\rightarrow Kl\nu$ Zerfalls}
 \end{figure}
\end{minipage}
\end{frame}

\begin{frame}
 \frametitle{Feynmangraph}
 \setcounter{framenumber}{5}
\begin{minipage}[h]{0.38\textwidth}
 \begin{enumerate}
  \item ruhendes $D$-Meson
  \item propagiert in $t$.
  \item $c$ wandelt unter Abstrahlung von $W^+$ in $s$
  \item $W^+$ zerstrahlt in Leptonpaar $l^+$, $\nu_l$
 \end{enumerate}
\end{minipage}
\begin{minipage}[h]{0.58\textwidth}
  \begin{figure}[h]
  \includegraphics[width = 1.9\textwidth]{../Abbildungen/DFeyn4.png}
   \caption{Feynmangraph des $D\rightarrow Kl\nu$ Zerfalls}
 \end{figure}
\end{minipage}
\end{frame}

\begin{frame}
\frametitle{\"Uberblick}
 \begin{minipage}[h]{0.62\textwidth}
 \begin{itemize}
  \item Fermis Goldene Regel: $\underbrace{\Gamma}_{\text{Rate}} = 2\pi \underbrace{\delta(E_f-E_i)}_{\text{Phasenraum }\Phi}\, \cdot \,{\underbrace{|\langle f|V|i\rangle|}_{\text{Amplitude }M}}^2$
  \item Teilchenstr\"ome
  \begin{itemize}
   \item [$\circ$] relativistischer Dirac-Strom
   \item [$\circ$] kurze Reichweite von $W^+$ f\"ur geringe Energien
   \item [$\rightarrow$] Beschreibung durch 4-Fermionen-Wechselwirkung
  \end{itemize}

  \item Starke WW zwischen $c$ und $\bar q_1$
  \begin{itemize}
   \item[$\circ$] erh\"alt Parit\"at $\mathcal{P}$
   \item[$\circ$] st\"orungsrechnerisch nicht erfassbar
   \item[$\rightarrow$] Darstellung durch \textbf{Formfaktoren} $f$
  \end{itemize}
 \end{itemize}
\end{minipage}
\begin{minipage}[h]{0.34\textwidth}
  \begin{figure}[h]
  \includegraphics[width = 1.3\textwidth]{../Abbildungen/DFeynSpez.png}
%    \caption{Feynmangraph des $D\rightarrow Kl\nu$ Zerfalls}
 \end{figure}
\end{minipage}
\end{frame}


\section{Fermis Goldene Regel}
\begin{frame}
\tableofcontents[currentsection]
\end{frame}
\subsection{Differentielle Zerfallsbreite d$\Gamma$}
\begin{frame}
- was ist gamma (wie wird es bestimmt, was hat es mit D* auf sich)
\end{frame}

\begin{frame}
\frametitle{Ergebnis der differentiellen Zerfallsbreite}
Fermis Goldene Regel:\\
\begin{align}
\mathrm{d} \Gamma(D\rightarrow Kl\nu) &= \frac{|M|^2}{2m_D} \mathrm{d}\Phi(K,\,l,\,\nu) \nonumber\\
&= \frac{G_F^2 |V_{cs}|^2}{24\pi^3}|f_+(q^2)|^2|p_K|^3\mathrm{d}q^2
\label{eq_theoGamma}
\end{align}
wird im Folgenden hergeleitet.
\vspace{0.5cm}

\small{Fermikonstante $G_F$,\\ CKM-Element $V_{cs}$,\\ Formfaktor $f_+$,\\Kaonimpuls $p_K$,\\ Impuls\"ubertrag $q^2$}
\end{frame}



\subsection{Phasenraumvolumen d$\Phi$}
\begin{frame}
\frametitle{Phasenraumvolumen}
-zweck
\end{frame}

\begin{frame}
 \frametitle{Berechnung}
 \begin{align}
 \mathrm{d} \Phi & = (2\pi)^4\frac{\mathrm{d}^3p_K}{2(2\pi)^3E_K} \frac{\mathrm{d}^3k_1}{2(2\pi)^3E_1} \frac{\mathrm{d}^3k_2}{2(2\pi)^3E_2}\delta^4(p_D-p_K-k_1-k_2)\nonumber\\
 &=\frac{1}{(2\pi)^5}\frac{\mathrm{d}^3p_K}{2E_K}\int \frac{\mathrm{d}^3k_1}{2(2\pi)^3E_1} \frac{\mathrm{d}^3k_2}{2(2\pi)^3E_2}\delta^4(q-k_1-k_2)k_{1,\mu} k_{2,\nu}\nonumber
 \end{align}
Leptonimpulse $k_i$ integriert, da Verteilung der Zerfallsbreiten durch $\mathrm{d}q^2$ gemessen. Die Eintr\"age $k_{1,\mu}$ und $k_{2,\nu}$ stammen von $M$ (folgt im Anschluss)
\end{frame}

\begin{frame}
 \frametitle{Berechnung}
 F\"ur die weitere Berechnung werden folgende Gleichungen des $D$-Meson-Ruhesystems ben\"otigt:
 \begin{itemize}
 \item $ \frac{\mathrm{d}^3p_K}{2E_K} = 2\pi |p_K|\mathrm{d}E_K$
   \item $|p_K| = \frac{\sqrt{\lambda(m_D^2, m_K^2, q^2)}}{2m_D}$
   \item $\int \frac{\mathrm{d}^3k_1}{2(2\pi)^3E_1} \frac{\mathrm{d}^3k_2}{2(2\pi)^3E_2}\delta^4(q-k_1-k_2)k_{1,\mu} k_{2,\nu} = \frac{\pi}{24}(q^2g_{\mu\nu} + 2q_\mu q_\nu)$
 \end{itemize}
 \vspace{0.7cm}
 \small{$\lambda = a^2+b^2+c^2-2(ab+bc+ac)$ - K\"all\'en-Funktion,\\$g_{\mu,\nu} = \text{diag}(1,-1,-1,-1)$ - Minkowski-Metrik}
\end{frame}

\begin{frame}
 \frametitle{Berechnung}
 Diese f\"uhren zu einem Ausdruck f\"ur das Phasenraumvolumen:
 \begin{align}
  \mathrm{d}\Phi = \frac{\pi}{24}(q^2g_{\mu\nu} + 2q_\mu q_\nu)\frac{|p_K|}{(2\pi)^4}\mathrm{d}E_K
  \label{eq_theoPhase}
 \end{align}
\end{frame}

\subsection{Matrixelement $M$}
\begin{frame}
\frametitle{Matrixelement}
- zweck
- berechnung an der tafel (grob)
- detaillierter auf folie
\end{frame}

\begin{frame}
 \frametitle{Berechnung I}
 hiernach verlinkung zu teilchenstroeme
\end{frame}

\begin{frame}
 \frametitle{Berechnung II}
 Abschliessend kann das quadrierte Matrixelement nach Umformung des leptonischen Anteils durch Casimirs Trick wie folgt geschrieben werden als
 \begin{align}
  |M|^2 & = \frac{G_F^2|V_{cs}|^2}{2}|f_+(q^2)|^2 P^\mu P^\nu \cdot 8\big(k_{l,\mu} k_{\nu,\nu} - g_{\mu\nu}k_lk_\nu + k_{l,\nu}k_{\nu,\mu}\big)\nonumber\\
  & = 4G_F^2|V|^2 |f_+(q^2)|^2 \big(2P^\mu P^\nu - P^2 g^{\mu\nu}\big) k_{l,\nu}k_{\nu,\mu}
  \label{eq_theoAmplitude}
 \end{align}
\end{frame}

\begin{frame}
\frametitle{Berechnung II}
 Nun k\"onnen \eqref{eq_theoPhase} und \eqref{eq_theoAmplitude} unter Verwendung von
 \begin{align*}
 \big(2P^\mu P^\nu - P^2 g^{\mu\nu}\big)\big(2q_\mu q_\nu + q^2g_{\mu\nu}) = 4 \lambda(m_D^2,m_K^2,q^2) = 16 m_D^2 |p_K|^2
 \end{align*}
 miteinander verkn\"upft und zu \eqref{eq_theoGamma} zusammengefasst werden
 ++Verlinkung auf f++

\end{frame}




\section{Teilchenstr\"ome}
\begin{frame}
\tableofcontents[currentsection]
\end{frame}

\subsection{Dirac-Gleichung}
\begin{frame}
\frametitle{Dirac-Gleichung}
\begin{itemize}
 \item Lorentzinvariant
 \item Für Spin $\sfrac12$ -Teilchen
 \item Besitzt positiv definite Wahrscheinlichkeitsdichte $j^0$
\end{itemize}
Dirac-Gleichung:
\begin{align}
 (i\gamma_\mu\partial^\mu-m)\psi = (i\slashed{\partial}-m)\psi=(\slashed{p}-m)\psi = 0
 \label{eq_dirac}
\end{align}
\vspace{0.5cm}

Dirac-Matrix $\gamma^\mu$,\\
Dirac-Wellenfunktion $\psi$,\\
Dirac-Spinoren $u$, $v$
\end{frame}

\begin{frame}
 \frametitle{Dirac-Strom $j^\mu$}
 \begin{itemize}
  \item Beschreibt Wahrscheinlichkeitsstrom eines propagierenden Teilchens
  \item Strom genügt Kontinuitätsgleichung $\partial_\mu j^\mu = 0$
 \end{itemize}

 \begin{align*}
  j^\mu = \bar \psi \gamma^\mu \psi. 
 \end{align*}

\end{frame}


\subsection{4-Fermionen-Wechselwirkung}
\begin{frame}
\frametitle{Rechtfertigung}
 \begin{minipage}[h]{0.58\textwidth}
\begin{itemize}
 \item Am $W^+$ koppeln hier ein hadronischer  und ein leptonischer Strom
 \item Hohe Masse des $W^+$ ($\approx$ 82 GeV) heißt kurze Lebensdauer
 \begin{itemize}
 \item[$\rightarrow$] Vier Fermionen(-ströme) wechselwirken in einem Punkt
 \end{itemize}
 \item Niederenergetischer Grenzfall ($q^2<2\GeV^2$) der GSW-Theorie
\end{itemize}
 \end{minipage}
 \begin{minipage}[h]{0.38\textwidth}
 \begin{figure}[h]
  \includegraphics[width = 3\textwidth]{../Abbildungen/4FermiPraes.png}
 \end{figure}
 \end{minipage}
 \end{frame}

 \begin{frame}
  \frametitle{Strom-Strom-Kopplung}
  \begin{itemize}
   \item Das Verhalten unter Lorentz-Transformationen ist für Ströme vielfältig (S, P, V, A, T)
   \item Aus Strömen konstruierter Hamiltonian muss (pseudo-)skalar sein (z.B. bei V-A-Kopplung)
   \item Diese erfordert \textbf{Paritätsverletzung} (Schwache WW koppelt an linkshändige Teilchen und rechtshändige Antiteilchen)
   \item Projektionsoperator $P=(1-\gamma_5)$ extrahiert linkshändige Komponente der Spinoren
   \begin{itemize}
    \item [$\rightarrow$] Dirac-Ströme wird um Axialstromanteil erweitert:
   \end{itemize}
  \end{itemize}
  \begin{align}
   j^\mu = \bar \psi \gamma^\mu (1-\gamma_5) \psi
  \end{align}
 \end{frame}

\section{Formfaktoren}
\begin{frame}
\tableofcontents[currentsection]
\end{frame}
\begin{frame}
 \frametitle{Motivation}
 \begin{itemize}
  \item Fermi-Wechselwirkung berücksichtigt die starke WW zwischen $c$ und $\bar q_1$ nicht
  \begin{itemize}
   \item [$\rightarrow$] gilt aber für Leptonenstrom
   \item [$\rightarrow$] Hadronenstrom durch Formfaktoren darstellen
  \end{itemize}
  \item Formfaktoren sind einheitenlose Größen, die theoretisch unzugängliche Einflüsse enthalten (sollen berechnet werden)
  \item Viererimpulse $p_D$ und $p_K$ sind einzige Freiheitsgrade und müssen zur Darstellung ausreichen
  \item Da QCD Parität erhält, müssen Formfaktorausdrücke dasselbe Transformationsverhalten unter Parität haben, wie $V$ bzw. $A$.
 \end{itemize}
\end{frame}

\subsection{Axialvektorformfaktoren und $f_-$}
\begin{frame}
\frametitle{Axialvektorformfaktoren}
\begin{itemize}
 \item Eigenwerte der Parität $\mathcal{P}$ sind $\pi = \pm1$ und multiplikativ, da diskrete Symmetrie
 \item Vektoren und Pseudoskalare transformieren mit $\pi = -1$, Axialvektoren mit $\pi = +1$
\end{itemize}
\begin{align}
 \mathcal{P} \, \big\langle\bar K^0\,\big|V^\mu|\,D^+\big\rangle &= (-1)\cdot(-1)\cdot(-1) = -1 \nonumber \\
 \mathcal{P} \, \big\langle\bar K^0\,\big|A^\mu|\,D^+\big\rangle &= (-1)\cdot(+1)\cdot(-1) = +1. \nonumber
\end{align}
\begin{itemize}
 \item Keine Kombination aus $p_D^\mu$, $p_K^\mu$ und $\epsilon^{\mu\nu\alpha\beta}$ transformiert mit $\pi = +1$
 \begin{itemize}
  \item [$\rightarrow$] $\big\langle K(p_K)\,\big|A^\mu\big|\, D(p_D)\big\rangle = 0$
  \item [$\rightarrow$] Keine Axialvektorformfaktoren!
 \end{itemize}
\end{itemize}
\vspace{0.7cm}
\small{Levi-Civita-Tensor $\epsilon^{\mu\nu\alpha\beta}$}

\end{frame}

\begin{frame}
 \frametitle{Vektorformfaktoren}
Viererimpulse selbst transformieren unter Parität wie Vektoren
  \begin{itemize}
   \item [$\rightarrow$] Allgemeine Darstellung durch zwei Formfaktoren $f_+$, $f_-$:
  \end{itemize}

\begin{align*}
 \big\langle K(p_K)\,\big|V^\mu\big|\, D(p_D)\big\rangle = f_+(q^2)(p_D+p_K)^\mu + f_-(q^2)(p_D-p_K)^\mu.
\end{align*}
\end{frame}

\begin{frame}
 \frametitle{Formfaktor $f_-$}
 Betrachtung von $M_-$ nur mit $f_-$:
 \begin{align}
  M_- &= \frac{G_F V_{cs}}{\sqrt{2}} f_-(q^2)(p_D-p_K)^\mu\bar u_\nu \gamma_\mu(1-\gamma_5)v_l\nonumber\\
 &= \frac{G_F V_{cs}}{\sqrt{2}} f_-(q^2)(k_\nu+k_l)^\mu \bar u_\nu \gamma_\mu(1-\gamma_5)v_l\,\nonumber\\
  &=\frac{G_F V_{cs}}{\sqrt{2}} f_-(q^2)\bar u_\nu (\slashed{k}_\nu + \slashed{k}_l) (1-\gamma_5)v_l\nonumber\\
 &\stackrel{\eqref{eq_dirac}}{=}\frac{G_F V_{cs}}{\sqrt{2}} f_-(q^2)\bar u_\nu (m_\nu + m_l) (1-\gamma_5)v_l\,.\nonumber 
 \end{align}
Die Leptonmassen sind für $l=e,\,\mu$ verglichen mit $m_D$ vernachlässigbar
\begin{itemize}
 \item [$\rightarrow$] $f_-$ liefert ebenfalls keinen Beitrag!
\end{itemize}
++link zu Matrixelement II++

\end{frame}


\subsection{Formfaktor $f_+$}
\begin{frame}
 \frametitle{Kinematische Grenzen}
\end{frame}

\begin{frame}
\frametitle{Parametrisierung}
parametrisierung
fit
resultate
\end{frame}

\begin{frame}
 \frametitle{Methode der kleinsten Quadrate}
\end{frame}

\section{Resultate unter $t_0$- und $\mathcal{O}$-Variation}
\begin{frame}
\tableofcontents[currentsection]
\end{frame}



\section{Ausblick}
\label{sec_ausblick}
\begin{frame}
\tableofcontents[currentsection]
\end{frame}


\begin{frame}
\frametitle{Bonus}
\textbf{Bonusfolien}
\end{frame}



\end{document}
